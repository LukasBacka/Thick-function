\section{}

\subsection{Introduction}

Tetration, or iterative tetration, is the fourth hyperoperation \cite{24} 
after addition, product and exponential function. Tetration is defined as

\[^nr := \underbrace{r^{r^{\iddots^{r}}}}_{\text{n}} \]

respectively copies of r are exponentially n times, we define it recursively

\begin{align}
        ^n r :=\left\{ 
        \begin{array}{ll}
                1 & (n = 0) \\
                r^{\left( ^{n-1}r \right)} & (n \geq 1)
        \end{array} \right.
\end{align}

apparently \(^{n-1}r = \log_r(^n r)\). 

The parameters r and a are referred to as basis and height. In this paper,
we will use special notation to express tetration.

\begin{align}
^n r:= \mathcal{T}(\underbrace{r, r, ..., r}_{\text{n}} | 1) := \exp_r^n 1
\end{align}

for simplicity we will write the vector in the form \((r_1, r_2, ..., r_x)\) 
as \(\xi_x\). We can consider this vector as an ordered x-tuple, for which we
create a more general notation \cite{28} \[\xi_k: = \bigotimes_{i=1}^k a_i \]
then we define the subtraction of the elements in \(\xi\) vector

\begin{definition}[subtraction of the elements in \(\xi\) vector]
        Let us have \(\xi_k\), where \(\xi_k :=\bigotimes_{i=1}^k a_i := (a_1, a_2,
        ..., a_k)\), for \(a_1, a_2, ... \in \mathbb{C}\). Subtraction of the elements 
        \(a_n\) from thr \(\xi\) vector, where \(k \geq n \geq 1\) we define:
        
        \begin{align}
                \xi_k \setminus a_n &:= \bigotimes_{\{k \geq j \geq 1\} \setminus n} a_i
        \end{align}
\end{definition}

We will use special notation for the most important elements: \(a_1 = \alpha\), 
\(a_2 = \beta\), \(a_3 = \gamma\), \(a_4 = \delta\). 
We will denote \(\omega\) the last element of the ordered xi vector. However, we
generalise \(\mathcal{T}\) - function and define it as follows:

\begin{definition}
\(\mathcal{T}\) function is recursive \(\mathcal{T}(n_1, n_2, ..., n_k | x) = 
n_1^{\mathcal{T}(n_2, ..., n_k | x)}\) with endpoints \(\mathcal{T}(n_k | x ) = n_k^x\) , 
where the coefficients  \((n_1, n_2, ..., n_k)\) make up the \(\xi\) vector,
 when it applies \(n_1, n_2, ..., n_k \in  \mathbb{C}\).  \(\mathcal{T}(n_1, 
 n_2, ..., n_k | x) \) is the \(\mathcal{T}\) k-th order function with base \(x\).
\end{definition}

Then it is necessary to define its inverse function, because the equation 
\(\mathcal{T}(2, 3 | x) = x\) we can reformulate into an equivalent 
form\(log_3(log_2(x)) = x\).

\begin{definition}[\(\mathcal{D}\)-function]
        The \(\mathcal{T}^{-1}\) function is a recursive function
        \begin{equation}
                \mathcal{T}^{-1}(n_1, n_2, ..., n_k | x) =
                log_{n_1}({\mathcal{T}^{-1}(n_2, ..., n_k | x)})
                \label{3}
        \end{equation}
        
with endpoints \(\mathcal{T}^{-1}(n_k | x ) = log_{n_k}(x)\), where 
\(n_1, n_2, ..., n_k \in  \mathbb{C}\). We choose notation 
\(\mathcal{T}^{-1}(\xi_k | x) := \mathcal{D}(\xi_k | x)\).
\(\mathcal{D}(n_1, n_2, ..., n_k | x) \) is \(\mathcal{D}\) 
k-th order function with base \(x\).

\end{definition}

\begin{definition}[Tau funkce]
        \(\tau\) function is the fixed point of function \(\mathcal{T}\) with the 
        \(\xi\) vector. This function is symmetric \(\overline{\tau(\xi_k)}=\tau(\overline{\xi_k})\).
        We can define \(\tau\) function as \[\mathcal{T}(\xi_k | \tau(\xi_k)) = \mathcal{D}(\xi_k | 
        \tau(\xi_k)) = \tau(\xi_k)\] where \(\xi_k = (n_1, ..., n_k); n_1, ..., n_k \in 
        \mathbb{C}\). \(\tau\) features in the shape of \(\tau(\xi_k)\) is cycle of kth order. 
        Values of tau function for \(\xi_k\) are neutral elements for the \(\mathcal{T}\)function with vector \(\xi_k\). 
        Subsequently, the following relations can be derived for the \(\tau\) function

        \begin{align}
                \tau(\xi_k) &= \frac{ln(\mathcal{D}(\xi_{k-1}|\tau(\xi_k))}{ln\|\omega\| + iArg(\omega)} & \\
                \tau(\xi_k) &= \alpha^{\frac{\mathcal{T}(\xi_k \setminus \alpha | \tau(\xi_k))}{2}}= 
                \|\alpha\|^{\frac{\mathcal{T}(\xi_k \setminus \alpha | \tau(\xi_k))}{2}} e^{i\mathcal{T}(\xi_k
                \setminus \alpha | \tau(\xi_k))Arg(\alpha)} 
                \label{4}
        \end{align}

\end{definition}
